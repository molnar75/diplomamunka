\Chapter{Összegzés}

A dolgozat során a szürkeárnyalatos képek kiszínezését vizsgáltam a Python eszközkészletének a segítségével. A vizsgálat első lépése a szegmentálás volt, a második pedig a szegmentált részek kiszínezése.

A kutatásom során több szakirodalmat megvizsgálva azt tapasztaltam, hogy a legelterjedtebb módszer a szegmentálás során az nem más, mint a K-means módszer. A módszer sikeressége főként abból adódik hogy egyszerűen, könnyen megvalósítható és használható, emellett a futási ideje is nagyon gyors. A tanulmányok többsége színes képekre használta ezt a módszer, szürkeárnyalatos képekre nem igazán találtam példát, így nem tudtam, milyen eredménnyel fog szolgálni, de mivel szimpatikus és jól dokumentált módszer, így emellett döntöttem. Már a vizsgálatom elején kiderült, hogy a képeket ha intenzitás szerint vizsgálja, nincsen elegendő információja ahhoz, hogy a képen el tudja különíteni az objektumokat. Próbálkoztam az értékek normalizálásával, standardizálásával, de sajnos ezek sem eredményeztek pontosabb szegmentálást.

Ezután tértem át a textúra alapú szegmentálásra, amiről úgy gondoltam, hogy megkönnyíti majd a K-means módszer dolgát. Ablakokat vágtam ki a képekből, és azokat adtam át magának a módszernek. Sajnos csalódnom kellett, mivel a K-means ezekből a feature vektorokból sem kapott elegendő információt ahhoz, hogy más eredményt szolgáltasson mint a sima, intenzitás alapú szegmentálás. Próbálkoztam ennél a módszernél is a feature vektorok normalizálásával, standardizálásával, azzal, hogy hozzáadok plusz információkat, például az intenzitások átlagát, de ezek a vizsgálatok sem javították a klaszterezés minőségét.

A szegmenseket ezzel a módszerrel nem tudtam pontosabban meghatározni, ezért áttértem egy másik módszer, a szuperpixel vizsgálatára. Ebben a témakörben 4 módszert is leteszteltem, amelyek közül 2 módszer elég pontosan meg tudta határozni a különböző szegmenseket. Sajnos ezekre a vizsgálatokra nem maradt elegendő időm, így a szuperpixelek meghatározása után a pixeleket egyből a színezésnek adom át, így a színezési eredmények nem a legpontosabbak. Ezt további, esetleg a szuperpixelek egyéb szegmentálásával lehetne fejleszteni. 

A szegmensek vizsgálata után áttértem a színezés tesztelésére. Itt két színteret is megvizsgáltam, az RGB és HSV színtereket. Színt adni a képnek nem volt nehéz, mivel csak egyszerű szorzásokat kellett elvégeznem a kép pixeleire. A kihívást az jelentette, hogy olyan színt adjak az adott szegmenseknek, ami az eredeti képre hasonlít. Ehhez konvolúciós neurális hálót tanítottam be hasonló képekre, mint amikre a tesztelést végeztem. A színezés során a K-means módszerrel szegmentált képeknél az objektumoknak többnyire ugyanazt a színt adta, attól függetlenül, hogy az objektumon belül több szegmens is megtalálható volt. A szuperpixel módszerek által szegmentált képekre a színezés bár változatosabb volt, de annál pontatlanabb. Az eredeti színeket csak kis százalékkal találta el mind a két esetben.

Összegezve a kutatásomat, a szürkeárnyalatos képekből visszanyerni a kép eredeti színeit az nem egy egyszerű feladat abban az esetben, ha nincsen a programnak kiindulási alapja. Amennyiben megadtam volna egy színtérképet, vagy ha egy videó képkockáit színezném ki, ahol 1 képkockát kézzel megadok, az algoritmus dolga pedig annyi lenne, hogy az alapján kiszínezi a többi képkockát, akkor a probléma sokkal egyszerűbb lenne. A szürkeárnyalatos képek nagyon kevés információt tartalmaznak, így önmagukban nem elegendőek ahhoz, hogy a vissza színezett kép tökéletesen megegyezzen az eredeti képpel.

\Chapter{Summary}

In my thesis work, I have examined the colorization of gray scale images by using the tools of Python.
The first main step is the estimation of the image segments, the second is the colorization of them.

After reviewing the literature of my research, I found that the most common method in segmentation is the K-means method. The success of the method is mainly due to the fact that it is simple, easy to implement and use, and its runtime is very fast. Most of the studies used this method for color images, I did not find many relevant examples for grayscale images, so I did not know what the results would be, but since it seems to be a suitable and well-documented method, therefore I chose it for the further research. Already at the beginning of my study, it turned out that if you examine the images by intensity, there is not enough information to be able to separate the objects in the image. I tried to normalize and standardize the values, but unfortunately these did not result in more accurate segmentation either.

I have tried the texture-based segmentation, which I thought would make the application of K-means method easier. I cut out windows from the images and passed them on to the method itself. Unfortunately, I had to be disappointed because K-means did not get enough information from these feature vectors to provide a different result than smooth, intensity-based segmentation. I also tried to normalize and standardize the feature vectors in this method by adding extra information, such as the average of the intensities, but these studies did not improve the quality of the clustering either.

I could not define the segments more precisely with this method, so I switched to another method, called superpixel. I also tested 4 methods in this topic, 2 of them were able to define the different segments quite accurately. Unfortunately, I do not have enough time left for these tests, so after defining the superpixels, I immediately pass the pixels to the coloring, so the coloring results are not the most accurate. This could be further developed, possibly with other segmentation of the superpixels.

After examining the segments, I switched to color testing. I also examined two color spaces here, the RGB and HSV color spaces. It was not hard to give color to the image because I just had to do simple multiplications on the pixels in the image. The challenge was to give the segments a color that resembled the original image. To do this, I configured and trained a convolutional neural network to images similar to those I tested. During the colorization, images segmented using the K-means method mostly gave the objects the same color, regardless of whether there were multiple segments within the object. For images segmented by superpixel methods, the coloring was more varied but more inaccurate. It were able to found only a small percentage of the original colors in both cases.

Summarizing my research, estimation of the original colors of an image from grayscale images is a difficult task if there is no starting point for the program. If I had specified a color map, or if I were coloring the frames of a video where I enter 1 frame manually and the algorithm would have to color the other frames based on it, the problem would be much simpler. Grayscale images contain very little information, so they alone are not enough to make the back-colored image look exactly like the original image.
