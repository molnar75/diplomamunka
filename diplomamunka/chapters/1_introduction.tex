\Chapter{Bevezetés}

Ha nekünk nem is, de nagyszüleinknek bizonyára megtalálhatók még régi, szürkeárnyalatos fényképek a fiók mélyén. Az ilyen régi, családi képeket nézegetve keltette fel az érdeklődésemet a dolgozat témaköre. Érdekelt, hogy vajon hogyan tudnánk az ilyen fotókat életre kelteni színek segítségével.

A képek kiszínezése két problémakörre osztható. 
Az első problémakör azon részek meghatározása a képen, amelyek eredetileg azonos színűek lehettek. Ha mi, emberek ránézünk egy képre, ezt könnyen meg tudjuk határozni, viszont a számítógép számára ez már nem ilyen egyszerű. A dolgozatban több módszert is megvizsgálok annak érdekében, hogy minél pontosabban tudjam meghatározni az azonos színű területeket.

A második problémakör az nem más, mint hogy hogyan tudjuk kiszínezni ezeket az összefüggő részeket, és meghatározni azt a színt, amely a legközelebb állhat az objektum eredeti színéhez. Mint az első problémakörnél, nekünk, embereknek ez szintén nem jelent kihívást, hiszen ha látunk például egy banánt egy szürkeárnyalatos képen, tudjuk, hogy valószínűleg sárga színű lehetett. A számítógépnek viszont meg kell tanítanunk, hogy adott tárgyak, adott textúrák milyen színűek, és ebből megpróbálja kitalálni hogy a szürkeárnyalatos képen szereplő tárgy, textúra milyen színű lehet.

A dolgozat elején bemutatásra kerülnek a probléma megoldásához használt általános eszközök. Azt követően a K-means klaszterező eljárás segítségével láthatjuk, hogy hogyan lehet a képet szegmensekre, régiókra bontani. Ezt a szegmentálás egy érdekes megközelítéseként a szuper pixeles vizsgálatok követik. Végül a képek egyes részeinek színére vonatkozó becslési eljárások következnek.
