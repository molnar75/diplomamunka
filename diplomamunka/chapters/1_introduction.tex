\Chapter{Bevezetés}

Ha nem is nekünk, de nagyszüleinknek bizonyára találhatók meg régi, szürkeárnyalatos fényképek a fiók mélyén. Az ilyen régi, családi képeket nézegetve keltette fel az érdeklődésemet a témakör. Érdekelt, hogy vajon hogyan tudnánk az ilyen fotókat életre kelteni színek segítségével.

A képek kiszínezése két problémakörre osztható. 

Az első problémakör azon részek meghatározása a képen, amelyek eredetileg azonos színűek lehettek. Ha mi, emberek ránézünk egy képre, ezt könnyen megtudjuk határozni, viszont a számítógép számára ez már nem ilyen egyszerű. A dolgozat során több módszert is megvizsgálok annak érdekében, hogy minél pontosabban tudjam meghatározni az azonos színű területeket a képen.

A második problémakör az nem más, mint hogy hogyan tudjuk kiszínezni ezeket az összefüggő részeket, és hogyan tudjuk meghatározni azt a színt, amely a legközelebb állhat az objektum eredeti színéhez. Mint az első problémakörnél, nekünk, embereknek ez szintén nem jelent kihívást, hiszen ha látunk például egy banánt egy szürkeárnyalatos képen, tudjuk, hogy valószínűleg sárga színű lehetett. A számítógépnek viszont meg kell tanítanunk, hogy adott tárgyak, adott textúrák milyen színűek, és ő ebből megpróbálja kitalálni hogy a megkapott szürkeárnyalatos tárgy, textúra milyen színű lehet.
