\Chapter{Klaszterezés}

Klaszterezésről pár dolog. \cite{clustering}

\Section{K-means klaszterezés}

\begin{python}
def kmeans_segmentation(image, k):
    """
    Segmenting the image using opencv-python's k-means method
    :param image: the image that I want to carry out the segmentation
    :param k: the number of the clusters
    :return: the compactness, the pixel_values and the labels
        from the k-means method
    """
    pixel_values = image.reshape((-1, 1))
    pixel_values = np.float32(pixel_values)

    criteria = \
        (cv2.TERM_CRITERIA_EPS + cv2.TERM_CRITERIA_MAX_ITER, 100, 0.2)

    compactness, labels, (centers) = cv2.kmeans(
        pixel_values,
        k,
        None,
        criteria,
        10,
        cv2.KMEANS_RANDOM_CENTERS)

    labels = labels.flatten()

    return compactness, pixel_values, labels
\end{python}

\Section{Optimális klaszterszám meghatározása}

Az optimális klaszterszám meghatározásához 4 módszert teszteltem le, ezeknek az eredménye a következő alfejezetekben található.
A módszereket a \cite{tomatoleaf} és a \cite{elbow} kutatások alapján választottam.

\SubSection{Silhouette módszer}

\[ S(k)=\frac{1}{num} \sum_{i=1}^{num} \frac{b(i)-a(i)}{maxa(i),b(i)}, \quad k \in n  \quad \cite{tomatoleaf} \]

\SubSection{Davies-Bouldin módszer}

\[ DB(k)=\frac{1}{k} \sum_{i=1}^{K} max \left(\frac{W_i + W_j}{C_{ij}}\right)  \quad \cite{tomatoleaf} \]

\SubSection{Calinski-Harabasz módszer}

\[ CH(k)=\frac{trB(k)/(k-1)}{trW(k)/(n-1)} \quad \cite{tomatoleaf}\]

\SubSection{Elbow módszer}

Módszer, futási nehézségek.

\SubSection{Összegzés}

A méretek szerinti futási eredményeket a \aref{tab:size_runtimes}. táblázat tartalmazza.

\begin{table}[h]
\centering
\caption{Futási idők átlaga különböző módszerek esetén}
\label{tab:size_runtimes}
\medskip
\begin{tabular}{|l|c|c|c|c|}
\cline{2-5}
 \multicolumn{1}{c|}{} & \multicolumn{4}{c|}{Kép mérete} \\
 \hline
 Módszer & 64px & 128px & 256px & 512px \\
\hline
Silhouette módszer & 0.2684s & 4.4029s & 68.0697s & 1183.2937s \\
Davies-Bouldin módszer & 0.0071s & 0.0042s & 0.0096s & 0.0274s \\
Calinski-Harabasz módszer & 0.0031s & 0.001s & 0.0028s & 0.0120s \\
\hline
\end{tabular}
\end{table}

A klaszterszám szerinti futási eredményeket a \aref{tab:cluster_runtimes}. táblázat tartalmazza.

\begin{table}[h]
\centering
\caption{Futási idők átlaga különböző klaszterszámok és módszerek esetén, 256px képméretre}
\label{tab:cluster_runtimes}
\medskip
\begin{tabular}{|l|c|c|c|}
\cline{2-4}
 \multicolumn{1}{c|}{} & \multicolumn{3}{c|}{Klaszterek száma} \\
 \hline
 Módszer & 2 & 4 & 8 \\
\hline
Silhouette módszer & 63.1462s & 63.1254s & 62.2278s \\
Davies-Bouldin módszer & 0.0060s & 0.0083s & 0.0064s \\
Calinski-Harabasz módszer & 0.0026s & 0.0024s & 0.0028s \\
\hline
\end{tabular}
\end{table}
