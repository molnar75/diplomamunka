\Chapter{Összegzés}

A dolgozat során a szürkeárnyalatos képek kiszínezését vizsgáltam a Python eszközkészletének a segítségével. A vizsgálat első lépése a szegmentálás volt, a második pedig a szegmentált részek kiszínezése. 

A kutatásom során több szakirodalmat megvizsgálva azt tapasztaltam, hogy a legelterjedtebb módszer a szegmentálás során az nem más, mint a K-means módszer. A módszer sikeressége főként abból adódik hogy egyszerűen, könnyen megvalósítható és használható, emellett a futási ideje is nagyon gyors. A tanulmányok többsége színes képekre használta ezt a módszer, szürkeárnyalatos képekre nem igazán találtam példát, így nem tudtam, milyen eredménnyel fog szolgálni, de mivel szimpatikus és jól dokumentált módszer, így emellett döntöttem. Már a vizsgálatom elején kiderült, hogy a képeket ha intenzitás szerint vizsgálja, nincsen elegendő információja ahhoz, hogy a képen el tudja különíteni az objektumokat. Próbálkoztam az értékek normalizálásával, standardizálásával, de sajnos ezek sem eredményeztek pontosabb szegmentálást.

Ezután tértem át a textúra alapú szegmentálásra, amiről úgy gondoltam, hogy megkönnyíti majd a K-means módszer dolgát. Ablakokat vágtam ki a képekből, és azokat adtam át magának a módszernek. Sajnos csalódnom kellett, mivel a K-means ezekből a feature vektorokból sem kapott elegendő információt ahhoz, hogy más eredményt szolgáltasson mint a sima, intenzitás alapú szegmentálás. Próbálkoztam ennél a módszernél is a feature vektorok normalizálásával, standardizálásával, azzal, hogy hozzáadok plusz információkat, például az intenzitások átlagát, de ezek a vizsgálatok sem javították a klaszterezés minőségét.

A szegmenseket ezzel a módszerrel nem tudtam pontosabban meghatározni, ezért áttértem a színezés vizsgálatára. Itt két színteret is megvizsgáltam, az RGB és HSV színtereket. Színt adni a képnek nem volt nehéz, mivel csak egyszerű szorzásokat kellett elvégeznem a kép pixeleire. A kihívást az jelentette, hogy olyan színt adjak az adott szegmenseknek, ami az eredeti képre hasonlít. Ehhez konvolúciós neurális hálót tanítottam be hasonló képekre, mint amikre a tesztelést végeztem. A színezés során az objektumoknak többnyire ugyanazt a színt adta, attól függetlenül, hogy az objektumon belül több szegmens is megtalálható volt. Az eredeti színeket csak kis százalékkal találta el. 

Összegezve a kutatásomat, a szürkeárnyalatos képekből visszanyerni a kép eredeti színeit az nem egy egyszerű feladat abban az esetben, ha nincsen a programnak kiindulási alapja. Amennyiben megadtam volna egy színtérképet, vagy ha egy videó képkockáit színezném ki, ahol 1 képkockát kézzel megadok, az algoritmus dolga pedig annyi lenne hogy az alapján kiszínezi a többi képkockát, akkor a probléma sokkal egyszerűbb lenne. A szürkeárnyalatos képek nagyon kevés információt tartalmaznak, így önmagukban nem elegendőek ahhoz, hogy a vissza színezett kép tökéletesen megegyezzen az eredeti képpel. 

\Chapter{Summary}

